\documentclass[12pt,oneside]{book}

\usepackage[english]{babel}
\usepackage[utf8]{inputenc}

\usepackage[a5paper]{geometry}

\pagestyle{plain}

\usepackage{hyperref}

\usepackage{amsmath}
\usepackage{amssymb}
\usepackage{amsthm}

\usepackage{csquotes}
\usepackage{cleveref}

\theoremstyle{definition}

\newtheorem{theorem}{Theorem}[chapter]
\newtheorem{lemma}[theorem]{Lemma}
\newtheorem{definition}[theorem]{Definition}
\newtheorem{proposition}[theorem]{Proposition}
\newtheorem{example}[theorem]{Example}
\newtheorem{corollary}[theorem]{Corollary}

\crefname{theorem}{Theorem}{Theorems}
\crefname{lemma}{Lemma}{Lemmas}
\crefname{definition}{Definition}{Definitions}
\crefname{proposition}{Proposition}{Propositions}
\crefname{example}{Example}{Examples}
\crefname{corollary}{Corollary}{Corollaries}

\begin{document}

\title{Logic}
\author{Pedro Minicz}
\date{}
\maketitle

\frontmatter

\tableofcontents

\mainmatter

%%%%%%%%%%%%%%%%%%%%%%%%%%%%%%%%%%%%%%%%%%%%
%%%   Set Theory: An Open Introduction   %%%
%%%%%%%%%%%%%%%%%%%%%%%%%%%%%%%%%%%%%%%%%%%%

\part*{Set Theory: An Open Introduction}
\addcontentsline{toc}{part}{Set Theory: An Open Introduction}
\setcounter{chapter}{7}
\renewcommand*{\theHchapter}{01\the\value{chapter}}

\chapter{The Iterative Conception}
Hello.


\chapter{Steps towards Z}
Hello.


\chapter{Ordinals}
Hello.


\chapter{Stages and Ranks}
Hello.


\chapter{Replacement}
Hello.


\chapter{Ordinal Arithmetic}
Hello.


\chapter{Cardinals}
Hello.


\chapter{Cardinal Arithmetic}
Hello.


\chapter{Choice}
Hello.


%%%%%%%%%%%%%%%%%%%%%%%%%%%%%%%%%%%%%%%%%%%%%%%%%%%%%%%%%
%%%   A Friendly Introduction to Mathematical Logic   %%%
%%%%%%%%%%%%%%%%%%%%%%%%%%%%%%%%%%%%%%%%%%%%%%%%%%%%%%%%%

\part*{A Friendly Introduction to Mathematical Logic}
\addcontentsline{toc}{part}{A Friendly Introduction to Mathematical Logic}
\setcounter{chapter}{0}
\renewcommand*{\theHchapter}{02\the\value{chapter}}

\chapter{Structures and Languages}
Hello.


\chapter{Deductions}
Hello.


\chapter{Completeness and Compactness}
Hello.


\chapter{Incompleteness from Two Points of View}
Hello.


\chapter{Syntactic Incompleteness---Groundwork}
Hello.


\chapter{The Incompleteness Theorems}
Hello.


\chapter{Computability Theory}
Hello.


\end{document}
